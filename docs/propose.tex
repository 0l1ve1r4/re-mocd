\documentclass{article}
\usepackage{graphicx}
\usepackage{amsmath}
\usepackage{hyperref}

\title{Project Proposal: Multi-Objective Community Detection in Social Networks}
\author{Guilherme Santos}
\date{\today}

\begin{document}
\maketitle

\section{Project Overview}
\begin{itemize}
    \item \textbf{Community Detection:} Using an Evolutionary/Genetic Algorithm (GA), where the adjacency list graph representation of the network and specific genetic operators are utilized to generate a Pareto front of candidate solutions.
    \item \textbf{Model Selection:} From the Pareto-optimal solutions, selecting the model based on predefined criteria.
\end{itemize}

\section{Project Phases and Timeline}

\begin{itemize}
    \item \textbf{Month 1: Review of State of the Art}
    \begin{itemize}
        \item Perform a literature review on existing multi-objective community detection algorithms.
        \item Study techniques involving evolutionary algorithms in community detection.
    \end{itemize}

    \item \textbf{Month 2-3: Study of Complex Network Concepts}
    \begin{itemize}
        \item Explore fundamental concepts of complex networks.
    \end{itemize}

    \item \textbf{Month 4-6: Model Construction}
    \begin{itemize}
        \item Develop the genetic algorithm for community detection, using adjacency lists to represent complex networks.
        \item Implement genetic operators:
        \begin{itemize}
            \item \textbf{Initialization:} Random generation of individuals, where each gene (node) randomly connects to one of its adjacent nodes.
            \item \textbf{Crossover:} Use uniform two-point crossover on loci based on adjacency representation.
            \item \textbf{Mutation:} Select random genes and reassign them to other randomly selected nodes.
        \end{itemize}
        \item Define the objective function:
        \[
        Q(c) = 1 - \text{intra}(c) - \text{inter}(c)
        \]
        where intra(c) minimizes the intra-link strength, and inter(c) minimizes the inter-link strength between partitions.
    \end{itemize}

    \item \textbf{Month 7: Model Validation with Artificial Networks}
    \begin{itemize}
        \item Validate the constructed model using artificial networks with known community structures.
        \item Evaluate performance against multi-objective criteria.
    \end{itemize}

    \item \textbf{Month 8: Data Collection and Processing}
    \begin{itemize}
        \item Gather and preprocess real-world social network data.
        \item Construct adjacency lists and format data for community detection.
    \end{itemize}

    \item \textbf{Month 9-10: Model Selection and Final Validation}
    \begin{itemize}
        \item Implement model selection criteria:
        \begin{itemize}
            \item \textbf{S\textsubscript{MAX}(Q):} Selects the solution maximizing \( Q(c) = 1 - \text{intra}(c) - \text{inter}(c) \).
            \item \textbf{S\textsubscript{max-min}:} Selects based on maximum deviation, \( \arg\max(\text{dev}(C, CD)) \).
        \end{itemize}
        \item Conduct a comprehensive validation of the selected model with real-world data.
    \end{itemize}
\end{itemize}

\section{Technical Approach}
The community detection algorithm will be based on an Evolutionary/Genetic Algorithm (GA) with the following components:

\subsection{Representation}
The network will be represented using an adjacency list with locus-based adjacency representation to structure the genetic algorithm.

\subsection{Genetic Operators}
\begin{itemize}
    \item \textbf{Initialization:} Individuals are randomly generated. For each gene, assign it to a random adjacency node of node \( i \).
    \item \textbf{Crossover:} Uniform two-point crossover on loci is applied to mix parent genes based on adjacency structure.
    \item \textbf{Mutation:} Randomly selected genes are re-assigned to other randomly selected adjacent nodes.
\end{itemize}

\subsection{Objective Function}
The objective function used to evaluate the quality of a partition \( c \) is defined as:
\[
Q(c) = 1 - \text{intra}(c) - \text{inter}(c)
\]
where:
\begin{itemize}
    \item \textbf{intra(c):} Minimizes the intra-link strength within partitions.
    \item \textbf{inter(c):} Minimizes the inter-link strength between partitions.
\end{itemize}

\section{Tools}
This project will utilize two primary programming languages:

\begin{itemize}
    \item \textbf{GIT:} Using git as versioning tool at: \href{https://github.com/0l1ve1r4/mocd}{git/mocd}
    \item \textbf{C Language:} C will be used for developing the core algorithm due to its speed and portability.
    \item \textbf{Python:} Python will be employed for data analysis, including processing the Pareto front output and performing model selection.
\end{itemize}

\section{Expected Outcomes}
\begin{itemize}
    \item An efficient and fast algorithm.
    \item A validated model selection mechanism based on Pareto-optimal criteria.
    \item Empirical results on both artificial and real-world social networks.
\end{itemize}

\end{document}
